\chapter{Conclusions and policy implications}

Fascinated by the public debate and the clamour for upcoming disruptive tax policies, I decided to analyze two papers in the field of corporate taxation.

The first, "Do countries compete over corporate tax rates?" by \textcite{dev-loc-red-08}. With a theoretical model first, and with empirical results later, the authors argue that jurisdictions strategically interact in setting their corporate income tax rate, with the ultimate goal not to lose taxable corporate income (profits). Notably, this interaction occurs among countries where capital controls are lower, eventually exerting downward pressure on taxation levels. To achieve these results, Nash equilibrium tax rates are calculated, and reaction functions are constructed for the government's choice of tax levels. It closes an analysis of the regression results with robust standard errors, and a final consideration by which the role of a yardstick competition in the rates setting is ruled out. 

The long-lasting decline in tax levels sparked my curiosity about its effects on governments' fiscal revenues, and here, the article "Corporate tax revenues in OECD countries" by \textcite{clausing} comes into play. The author breaks down the government's fiscal revenues into four terms and highlights the role of firms' behaviour in determining tax revenues. In doing so, for her sample, she finds that the profit-maximizing corporate income tax rate is around 33\% and that, over time, the average corporate income tax revenues have remained almost constant thanks to policies that broadened the tax base.

Both discussions draw attention to the need for international tax coordination and harmonization efforts. An unregulated and irrational tax competition would sooner or later bring rates to the bottom and, therefore, in the short term, damage governments' fiscal budgets. As a matter of fact, in recent years, international organizations, such as the OECD, have strengthened their relationships with countries and supranational entities. A result of such activities is the upcoming Base Erosion Profit Shifting working project, which, among others, will introduce a breakthrough 15\% global minimum tax on the corporate income of given firms.