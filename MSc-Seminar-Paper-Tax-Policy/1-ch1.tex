\chapter{Introduction}

In the last few decades, the global landscape of corporate income taxation has witnessed a significant transformation characterized by a widespread decline in tax rates levied on corporate profits. For instance, in OECD countries, the average statutory tax rate on firms' income was 46.5\% in 1982 and reached 21.2\% in 2022. This phenomenon has sparked intense debate among policymakers and economists regarding its potential impact on tax revenues and broader economic implications. Understanding what is driving reductions in tax levels and the effects on government revenues is crucial for designing effective tax policies and ensuring sustainable fiscal frameworks, and is thus the motive behind this discussion.

As countries increasingly engage in cross-border trade and investments, the effectiveness of corporate income tax policies becomes intertwined with considerations of capital mobility and investment flows. Quinn's index on the intensity of capital controls indeed reveals a concomitant relaxation of capital restrictions to the rise of trade activities. Consequently, as countries become more interconnected through trade, the need for effective tax coordination becomes evident. In fact, harmonizing tax policies and promoting international tax cooperation can help address challenges such as base erosion, profit shifting, and tax avoidance, which can undermine fair taxation and erode tax revenues. The efforts of the OECD are heading in this direction, and a particular example is the BEPS project that started back in 2013. Aiming to tackle harmful base erosion and profit shifting, as of today, it is signed by 135 countries and, in December 2022, the Council of the European Union unanimously agreed to implement the Minimum Tax Directive, which mirrors Pillar Two of the BEPS project and introduces the global minimum tax on corporate profits.

Part of the literature, such as \textcite{weichenrieder}, in spite of the drastic downward trend in corporate tax rates, has investigated the advantages of retaining corporate taxation. Firstly, corporate income taxation serves as a vital source of tax revenues for governments. This is especially true for developing countries, where corporate tax revenues comprise a large part of the fiscal budget. For instance, in 2021, in Colombia and Chile, corporate income tax revenues accounted for 23\% and 17\% of all tax revenues, respectively, while in Germany and Italy they accounted for only 6\% and 4\%, respectively. Secondly, corporate income taxation acts as a backstop for personal income tax systems. Without corporate income taxation, there would be an increased incentive for individuals to reclassify their income as corporate profits, thereby reducing their personal tax liability. All in all, corporate income taxation still plays a central role in many different ways, and therefore downward tax rates must be carefully analyzed.

The following discussion breaks up as follows. Initially, the paper  "Do countries compete over corporate tax rates?" by \textcite{dev-loc-red-08} is described and commented. Here, the authors study how governments set their tax levels as part of a strategic interaction with other jurisdictions. Building up a decision model for firms' and countries' strategies, they conclude by arguing that the decline in corporate income tax rates can be attributed in large part to increased tax competition caused by the loosening of capital restrictions. It follows the second article, "Corporate tax revenues in OECD countries" by \textcite{clausing}. She examines the determinants of tax revenues, breaking them down into their components so that governments can understand what they are determined by. In fact, revenues are not only driven by tax rates, but the reaction of businesses, in terms of profitability or participation in the economy, also plays an important role. In addition, the third Section links the previous two analyses, poses criticisms and suggestions, and unveils an upcoming area for policy action. Lastly, a sum-up closes the essay.