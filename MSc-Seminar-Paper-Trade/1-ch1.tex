\chapter{Introduction}

Climate change and its consequences are one of the greatest challenges of our time. Human activities, with the emission of greenhouse gases, have unequivocally caused global warming \parencite{IPCC2023}.  To keep global warming within a manageable path, numerous mitigation and adaptation actions and policies are needed. If applied on a large scale, economic instruments can also help reduce emissions, and among these, carbon taxes and their various implementations play an important role. As the report of \textcite{IPCC2023} reveals, in 2020, over 20\% of global greenhouse gas emissions were covered by carbon taxes or Emission Trading Systems. These instruments have proven to be effective in reducing emissions and, in addition, carbon tax revenues are often redistributed to low-income households to address equity issues.

In this context, \textcite{Larch2017} study the impacts of introducing carbon tariffs on trade flows, welfare, and emissions. To do so, they implement a widely used framework in the international trade literature, such as a structural gravity model. Moreover, taking advantage of the work of \textcite{copeland2005}, they are also the first to isolate the effects that influence emissions in this framework. Not only this, \textcite{Larch2017} also assess the rate of carbon leakage that arises when only a subset of countries commit to their climate commitments, with and without carbon tariffs.

This seminar paper proceeds as follows: Chapter \ref{sec:theorethical} summarizes the paper of \textcite{Larch2017}, Chapter \ref{sec:literature} reviews the literature on carbon tariffs and compares results, Chapter \ref{sec:extension} proposes two extensions to the paper of \textcite{Larch2017}, Chapter \ref{sec:conclusion} concludes.