\chapter{Introduction}

What is behind the success of business training programs? The International Labour Organization (ILO) selects annual success stories of participants in its programs, here is one. Pinneti Lakshmi, married at 19 in a village in India, sought to increase her family's income. Initially starting a tea shop that eventually closed, she joined an entrepreneurship program by the local NGO. Completing the training, Lakshmi ventured into crafting and selling jute products, aided by memberships in local organizations and bank support. Her cottage industry now employs 10 people, each earning 250-300 Rupees daily. Looking ahead, Lakshmi plans to expand its production capacity and begin selling jute products on a national scale \parencite{ILO2022}. Stories like Lakshmi's reveal the important role of development programs in emerging countries. Business training programs such as ILO's "Start and Improve Your Business" can potentially transform lives within entire villages and unlock opportunities, ultimately enabling shared and inclusive development of local communities.

Micro and Small Enterprises (MSEs) in emerging economies are typically characterized by low productivity and survival rates, poor working conditions and profitability, and a low adoption of recommended business practices. Standard training programs aim to teach participants so-called "hard skills," such as financial management, marketing, inventory control, or quality control. Over the years, many international organizations have taught these topics worldwide, and researchers have begun to study the impact, effectiveness, and efficiency of these programs. They often found that their cost-benefit analysis was not convincing enough, for example, because their average effect on entrepreneurs' profits is between 5 and 10\% \parencite{McKenzie2020}. As a result, many alternative designs and extensions of these training programs have been proposed and tested. A first common extension is the inclusion of mentors who assist participants after training, as in the papers by \cite{Bakhtiar2022} or \cite{Campos2017}, the latter combining mentors and soft-skills. In fact, the psychological literature confirms that a proactive personality and persistent behavior are essential to the success of a business \parencite{Frese2014}. 

In this context, in "The impact of soft-skills training for entrepreneurs in Jamaica", \cite{Ubfal2022} implement a Randomized Control Trial (RCT) in Jamaica for two types of training (i.e., treatments): a combined one, which only briefly covers soft-skills, and the other that focuses more in-depth on topics such as problem solving, persistence, and learning from mistakes. Their experiment took place between 2016 and 2017, on a sample of 945 selected entrepreneurs, divided into three groups (control, combined training, soft-skills training). Before the treatment, participants completed a baseline survey, while after the classes they were asked to respond to two follow-up surveys, after 3 and 12 months. At this stage, the researchers experienced some attrition but still managed to obtain robust results showing a positive and significant effect of soft-skills training treatment on both short-term profits and the rate of adoption of business practices. Some critical findings of their experiment are the low persistence of treatment effects, detected only in the first follow-up, and the heterogeneous increase in business outcomes, experienced only by men. In contrast, entrepreneurs who underwent soft skills training were found to be more persistent even after 12 months. Overall, \cite{Ubfal2022} well describes the power of including soft-skills topics in business training, confirming their validity in enhancing their impact.

Chapter \ref{sec-2-summary} summarizes the paper by \cite{Ubfal2022}. Chapter \ref{sec-literature-rev} reviews the literature on business training programs and their extensions. Chapter \ref{sec-4-discussion} analyzes the strengths and weaknesses of the paper and discusses future extensions. Chapter \ref{sec-5-conc} concludes.