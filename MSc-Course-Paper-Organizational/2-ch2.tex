\chapter{The impact of soft-skills training for entrepreneurs in Jamaica }
\label{sec-2-summary}

Jamaica, an island in the Greater Antilles, is a Commonwealth realm with a population of over 2.8 million people. Since its independence in 1962, the country has experienced a modest growth mainly driven by the services and tourism sectors. Today, Jamaica registers a low unemployment rate of around 6\%, offset by the lowest ever total factor productivity \parencite{WorldBank}.

In this context, \cite{Ubfal2022} set up a Randomized Control Trial (RCT) to test the impacts (i.e., causal effects) of two different types of business training programs on small entrepreneurs and their business outcomes. The authors worked together with the Jamaica Business Development Corporation (JBDC), a governmental organization, which was in charge of delivering the training programs. This Section presents the paper and its findings, highlighting in particular the effectiveness of soft-skills training in fostering the adoption of recommended business practices.



\section{Methodology}

\subsection{Training Programs}

Together with JBDC, the research team developed and adapted for the local context two training programs, which represent the two treatments. Both were taught by the same teachers in the same facilities over 10 weeks, and both had the same duration, 40 hours, split into two parts, where the second half differentiated each program. The lectures took place between October and December 2016.

During the first common 5 weeks, the lectures aimed to develop personal initiative in the participants. A choice motivated by the notion that a proactive trait is essential for entrepreneurs to be able to anticipate and prepare for potential opportunities and threats \parencite{Frese2014}.

In the remaining weeks, the combined training expanded its contents to typical business training topics, such as financial management or marketing. Instead, the soft-skills training provided more materials related to personal initiative, such as perseverance and problem solving. This different approach, which includes insights from the psychology literature, is the real focus of the following analysis because the authors are interested in assessing whether practicing more the participants' soft-skills is more beneficial, in terms of business outcomes, than simply discussing business practices.

\subsection{Sampling and Randomization}

\begin{table}
\centering
\caption{Baseline Balance (extract), \cite{Ubfal2022}}
\label{sum-baseline-balance}
\begin{adjustbox}{width=\textwidth}
\begin{tabular}{lccccc}
\toprule
                                                                                                      & Control group (C) & Soft-skills training (T1) & Combined training (T2) & T1 = C               & T2 = C                \\
                                                                                                      & Mean              & Mean                      & Mean                   & P-val.               & P-val.                \\ 
\hline
\textbf{Panel A. Stratification variables}                                                            &                   &                           &                        &                      &                       \\
Female                                                                                                & 0.58              & 0.59                      & 0.58                   & 0.87                 & 1.00                  \\
Has employees                                                                                         & 0.30              & 0.30                      & 0.30                   & 0.86                 & 0.91                  \\
Education: more secondary                                                                             & 0.61              & 0.61                      & 0.60                   & 0.87                 & 0.81                  \\
\textbf{Panel B. Owner characteristics}                                                               &                   &                           &                        &                      &                       \\
Age                                                                                                   & 42.43             & 41.29                     & 42.20                   & 0.22                 & 0.81                  \\
Black                                                                                                 & 0.90              & 0.92                      & 0.91                   & 0.39                 & 0.53                  \\
Married                                                                                               & 0.46              & 0.39                      & 0.44                   & 0.06                 & 0.62                  \\
Set a goal for business                                                                               & 0.84              & 0.85                      & 0.84                   & 0.77                 & 0.81                  \\
Wants to channge sth                                                                                  & 0.64              & 0.63                      & 0.68                   & 0.92                 & 0.27                  \\
Personal initiative                                                                                   & 6.01              & 6.01                      & 6.07                   & 0.99                 & 0.32                  \\
\textbf{Panel C. Firm characteristics}                                                                &                   &                           &                        &                      &                       \\
Keeps formal accounts                                                                                 & 0.08              & 0.09                      & 0.13                   & 0.59                 & 0.03                  \\
Registered business                                                                                   & 0.53              & 0.51                      & 0.54                   & 0.72                 & 0.69                  \\
Sales in the last month                                                                               & 87766             & 100744                    & 75922                  & 0.47                 & 0.42                  \\
Business practice index                                                                               & 0.58              & 0.59                      & 0.61                   & 0.57                 & 0.18                  \\ 
\hline
\begin{tabular}[c]{@{}l@{}}\textbf{Panel D. Aggr. Orthog. test}\\\textbf{for~panels B-C}\end{tabular} &                   &                           &                        &                      &                       \\
P-value                                                                                               &                   &                           &                        & 0.91                 & 0.52                  \\ 
\hline
Observations                                                                                          & 315               & 315                       & 315                    & \multicolumn{1}{l}{} & \multicolumn{1}{l}{}\\
\bottomrule 
\end{tabular}
\end{adjustbox}
\end{table}

Entrepreneurs were recruited through a telemarketing campaign. At first, around 2000 business owners living in Kingston and surroundings expressed their interest in the program. Secondly, to gather information about the participants and their businesses, a baseline survey was given to the contacts. To reduce heterogeneity and create more comparable groups, the authors established certain eligibility criteria, such as having fewer than five employees and reporting less than JMD 1 million in monthly sales and costs, thus increasing the sensitivity of the analysis and drawing more reliable conclusions. The final sample included 945 entrepreneurs. Lastly, to ensure the RCT design, after a stratification based on certain variables like gender and education, business owners were randomly assigned to the two training programs, i.e. treatments, and to a control group.

Table \ref{sum-baseline-balance} summarizes the characteristics of the three groups, depicting a successful balanced randomization. In the three samples, female entrepreneurs are 58\%, the mean age is around 42 years old, and between 39\% and 46\% are married. The majority has a goal for its business (84\%) and wants to change something in it ($\sim$~64\%). On a scale of 1 to 7 (maximum), the personal initiative indicator scores 6. Half of the businesses are registered, but only a small portion keeps formal accounts ($\sim$ 9\%), and business practices are only partially widespread (59\%).

\subsection{Instruments}

To measure the impacts of the experiment on the business outcomes and a series of intermediate channels through which the treatments may interact, \cite{Ubfal2022} collected data during and after the trial with different surveys and tests.

The three main instruments are the baseline survey and the two follow-ups at 3 and 12 months, in order to capture potential short and long-term effects. The baseline survey was carried out during the sampling phase, between August and September 2016. Three months after the end of the training, in March 2017, the first follow-up survey was conducted  by a designated international survey firm. Questions related to sales, profits, business practices, and soft skills were asked. At this stage, the response rate stood at 73\% in total. In January 2018, the second follow-up took place. The same questions were asked, but the response rate stopped at 59\%. The authors explain the low participation rate in the subsequent poll by pointing out that lottery scams are very common in Jamaica and discourage people from participating in interviews, even when they are offered monetary incentives.

In addition to the ones already mentioned, class attendance was observed, and a knowledge test was given at the penultimate class. Finally, the second follow-up survey contained an additional set of questions aimed to assess specific soft-skills as well as another questionnaire designed to measure participants' grit.

\subsection{Statistical Analysis}

The RCT design chosen by \cite{Ubfal2022}, by randomly assigning participants to treatment and control groups, minimizes the selection bias and offers a robust method for determining causal relationships. In the following linear regression model, the coefficients of the treatment dummy variables represent the estimated treatment effects on the various business outcomes studied by the authors, such as firm survival, sales, and profit changes. The authors boost the statistical power using ANCOVA regressions and report intention-to-treat effects, analyzing outcomes based on initial training assignments, irrespective of actual participation. The following OLS regression is estimated:
\begin{equation}
Y_{it}=\alpha+\boldsymbol{\beta_1}T1_i+\boldsymbol{\beta_2}T2_i+\delta X_{i0}+\beta_0Y_{i0}+\epsilon_{it}
\end{equation}
where $Y_{it}$ is the outcome for individual $i$ at the first or second follow-up, $\alpha$ is the intercept, $T_i$ is the treatment dummy variable, $X_{i0}$ is a vector of baseline control variables, $Y_{i0}$ is the baseline outcome, and $\epsilon_{it}$ is the residual. The coefficients of interest are $\beta_1$ and $\beta_2$, indicating the change in the outcome variable based on the respective treatments. Heteroskedasticity-robust standard errors are used.

\vspace{-5px}

\section{Results}

\subsection{Attendance and Retention}

In line with the attendance rates of other experiments, about 80\% of participants attended at least one class, with 60\% attending the minimum required for the diploma, five classes. Furthermore, in an attempt to forecast the profile of effective participants, the authors regressed observed characteristics on participation in at least one class. Most of the features are insignificant, such as gender or internet access, however, older entrepreneurs with a goal for their company and who have registered their company are more likely to participate in the program.

During the second to last lecture, entrepreneurs were tested on their training materials and also on the other group's contents. Participants consistently scored higher on questions relating to the content presented in their course, indicating a good level of retention.

\vspace{-5px}

\subsection{Impact on Business Outcomes}
\label{sum-sec-business-outcomes}

Table \ref{sum-business-outcomes} shows the impact of the two trainings on three measures of business outcomes after 3 and 12 months.

Any statistically significant effect is observed in terms of survival. After 3 months, 81\% of the businesses in the control group still exist, and only an imprecise yet positive and temporary coefficient is observed for the soft-skill training. A first key finding is that, after 3 months, the soft-skills training leads to an observed 11\% increase in the likelihood of participants reporting positive profits, compared with the control group's baseline, 47\%. In addition, the highly positive coefficient of the sales and profits index provides additional evidence of soft skills training's short-term efficacy. However, in the second follow-up, neither of the two coefficients is significant, which leaves the authors wondering why training benefits aren't as persistent.

\begin{table}
\centering
\caption{Impacts on Business Outcomes (extract), \cite{Ubfal2022}}
\label{sum-business-outcomes}
\begin{adjustbox}{width=\textwidth}
\begin{tabular}{lcccccc} 
\toprule
                     & \multicolumn{2}{c}{Firm survival} & \multicolumn{2}{c}{Positive
  profits} & \multicolumn{2}{c}{Sales and
  profits index}  \\
                     & 3-months & 12-months              & 3-months & 12-months                   & 3-months & 12-months                           \\ 
\hline
Soft-skills training & 0.05     & -0.02                  & \textbf{0.11**}   & 0.00                        & \textbf{0.28**}   & -0.08                               \\
Combined training    & -0.03    & 0.01                   & 0.07     & -0.07                       & 0.13     & -0.08                               \\
Mean control group   & 0.81     & 0.93                   & 0.47     & 0.47                        & 0.00     & 0.00                                \\
\bottomrule
\end{tabular}
\end{adjustbox}
\end{table}

\vspace{-5px}

\subsection{Impact on Intermediate Outcomes}
\label{sum-sec-intermediate-out}

\begin{table}
\centering
\caption{Mechanisms (extract), \cite{Ubfal2022}}
\label{sum-mechanisms}
\begin{adjustbox}{width=\textwidth}
\begin{tabular}{lcccccccc} 
\toprule
                     & \multicolumn{2}{c}{Business
  practices} & \multicolumn{2}{c}{Personal
  initiative} & \multicolumn{2}{c}{Introduced
  innovation} & \multicolumn{2}{c}{Loan requested}  \\
                     & 3-months         & 12-months             & 3-months & 12-months                      & 3-months        & 12-months                 & 3-months & 12-months                \\ 
\hline
Soft-skills training & \textbf{0.09***} & 0.04                  & 0.09     & 0.14                           & \textbf{0.12**} & 0.05                      & 0.04     & \textbf{0.09*}           \\
Combined training    & 0.04             & 0.03                  & -0.03    & -0.13                          & 0.04            & 0.01                      & 0.04     & 0.05                     \\
Mean control group                     & 0.46             & 0.55                  & 0.00     & 0.00                           & 0.36            & 0.46                      & 0.08     & 0.33                     \\
\bottomrule
\end{tabular}
\end{adjustbox}
\end{table}

The results presented in Section \ref{sum-sec-business-outcomes} show the effects of the two training programs on business outcomes, particularly highlighting the importance of shaping and fostering the entrepreneurial mindset of the participants. Now, based on the answers of the two follow-ups, \cite{Ubfal2022} try to assess through which channels the trainings interacted, namely with which personal and business features they engaged the most. Table \ref{sum-mechanisms} provides the OLS coefficients.

A first surprising result is an increased likelihood of adoption of the recommended business practices by participants in the soft skills training, although only the combined group discussed them. This first statistically significant treatment effect tells us that 9\% more entrepreneurs, compared to a control group average of 46\%, adopted the recommended practices after three months. In contrast, the treatment effect of the combined training is smaller and not significant. The authors point out that this outcome is in line with the results of other experiments and come to a first conclusion, which is that stimulating the entrepreneurial mindset of business owners is more effective in promoting the adoption of business practices than merely discussing them.

Both trainings have no significant impact on the personal initiative index after either 3 or 12 months, although both had 5 lessons focusing on related topics. While this result was to be expected for the combined training, which might have a dilution effect, the opposite was expected for the soft-skills training, which indeed has a larger but not significant positive coefficient.

A second statistically significant result after 3 months is a wider introduction of innovations by soft-skills participants, 12\% more than the control group average of 36\%. Similarly to the business practices coefficient, both emerge in the first follow-up but disappear after 12 months. Once again, questioning the low persistence of such impacts could help find better program design that leads to more lasting intermediate and final results.

One last effect appears only in the second follow-up. After 12 months, there is a higher share of entrepreneurs from the soft-skills training requesting for a loan. Compared with a control group average of 33\%, only 5\% more business owners from the combined training applied for a loan, but this coefficient is not significant. Since topics related to financial management were discussed in their classes, which may have given them a knowledge advantage, a significant positive result was expected, but this is not the case.

\vspace{-5px}

\subsection{Impact on Soft Skills}

In addition to the consequences on business outcomes of business training, \cite{Ubfal2022} briefly analyze whether the inclusion of psychological insights in the design of such programs, which in their experiment results in having 5 introductory lessons on personal initiative and then two different treatments, can really have an impact on the mindset and personal behavior of entrepreneurs. Already in Section \ref{sum-sec-intermediate-out}, when examining intermediate mechanisms, the authors partially studied the effect of treatments on a personal initiative index but found no significant effect. Here, motivated by the notion from the psychological literature that a proactive and setback-resistant mindset is beneficial for entrepreneurs' success, \cite{Ubfal2022} examine the results of two surveys regarding soft-skills impacts attached to the second follow-up.

Table \ref{sum-soft-skills} shows in column (1) the outcomes of the first survey, which consisted of Likert-type questions on soft-skills topics taught throughout the program. As expected, soft-skills training has a positive impact on all measures and, in particular, a significant treatment effect on participants' perseverance, as this particular trait was discussed more with them. In contrast, the combined training, in which participants were given only 5 introductory soft-skills topics, does not exhibit any effect.

The second additional survey consisted of a game adapted from \cite{Alan2019}, designed to measure the perseverance of the treated participants. In each round (6 in total), participants had to choose whether to attempt an easy or difficult task, the second paying more. To give participants a sense of difficulty, the first round was easy and the second difficult for everyone. From the third round onward, after completing the task, participants could choose the difficulty. Table \ref{sum-soft-skills} column (2) shows, in order, how many participants chose the difficult task in each round and how many times the difficult task was chosen per participant compared to the control group, and finally an aggregate index. The probability for soft-skills participants to choose the difficult task in all rounds is 8\% points higher than the control group's average, 30\% points, and, in fact, they chose the difficult task significantly more times than the easy task, 0.32 times more than the control group's baseline value of 2.05. Overall, the task difficulty index appears positive and statistically significant for the soft-skills group, confirming both the effectiveness of soft-skills training in changing participants' behavior and mindset and the retention of materials (i.e., the quality of the program). Additionally, it is interesting to note that these additional surveys were able to measure the persistence after 12 months of the impact on entrepreneurs' mindset, in particular perseverance, which the standard follow-up did not.

\begin{table}
\centering
\caption{Impacts on Soft Skills (extract), \cite{Ubfal2022}}
\label{sum-soft-skills}
\begin{adjustbox}{width=\textwidth}
\begin{tabular}{lcccclccc} 
\toprule
                     & \multicolumn{4}{c}{(1)}                                            &  & \multicolumn{3}{c}{(2)}                             \\
                     & \multicolumn{4}{c}{Self-Reported}                                  &  & \multicolumn{3}{c}{Game}                            \\ 
\cline{2-5}\cline{7-9}
                     & Grit & Perseverance(APS) & Personal initiative & Soft skills index &  & All rounds & Num. of rounds & Difficult task index  \\ 
\hline
Soft-skills training & 0.16 & \textbf{0.22**}   & 0.14                & \textbf{0.14**}   &  & 0.08       & \textbf{0.32*} & \textbf{0.21*}        \\
Combined training    & 0.02 & -0.09             & -0.13               & -0.04             &  & 0.00       & 0.08           & 0.05                  \\
Mean control group   & 0.00 & 0.00              & 0.00                & 0.00              &  & 0.30       & 2.05           & 0.00                  \\
\bottomrule
\end{tabular}
\end{adjustbox}
\end{table}

\subsection{Treatment Effects by Gender}

Lastly, \cite{Ubfal2022} evaluate their results again, but this time they interact them with the gender of the entrepreneur. In this context, assessing treatment effects by gender is important because it helps researchers identify how men and women respond uniquely to training (contributing further to the literature), ultimately allowing for more effective, customized programs and a better (cost-effective) spending of the funding.

Table \ref{sum-gender} reveals clearly how the short-term effects on business performance are driven by men. On average, 19\% more male entrepreneurs in the soft-skills training reported positive profits after three months. However, when this measure is interacted with the dummy variable representing women (i.e., $D[female=1]$), the increase completely cancels out, -0.13, meaning the treatment did not impact their profits.

In addition, the authors examine the differentiated effects on the intermediate outcomes. A major change in the measures originates from men, however, a surprising key discovery lies in the fact that women who participate in soft-skills training also experience a positive change in their adoption of business practices, 0.02. This last finding leads to the conclusion that soft-skills training encourages both men and women to adopt business practices, but only for males does it produce short-term increases in business outcomes.

\begin{table}
\centering
\caption{Impacts on Business Outcomes and Mechanisms by Gender after 3 months (extract), \cite{Ubfal2022}}
\label{sum-gender}
\begin{adjustbox}{width=\textwidth}
\begin{tabular}{lccclcc} 
\toprule
                                                      & \multicolumn{3}{c}{\begin{tabular}[c]{@{}c@{}}(1)\\Business Outcomes\end{tabular}}                                                                                                    &  & \multicolumn{2}{c}{\begin{tabular}[c]{@{}c@{}}(2)\\Mechanisms\end{tabular}}                                                 \\ 
\cline{2-4}\cline{6-7}
                                                      & \begin{tabular}[c]{@{}c@{}}Firm\\survival\end{tabular} & \begin{tabular}[c]{@{}c@{}}Positive\\profits\end{tabular} & \begin{tabular}[c]{@{}c@{}}Sales and profits\\index\end{tabular} &  & \begin{tabular}[c]{@{}c@{}}Business\\practices\end{tabular} & \begin{tabular}[c]{@{}c@{}}Personal\\initiative\end{tabular}  \\ 
\hline
Soft-skills training                                  & 0.08                                                   & \textbf{0.19***}                                          & \textbf{0.66**}                                                  &  & \textbf{0.08*}                                              & 0.16                                                          \\
Combined training                                     & 0.03                                                   & \textbf{0.12*}                                            & 0.33                                                             &  & 0.04                                                        & 0.06                                                          \\
Soft-skills training $\times$ female & -0.06                                                  & -0.13                                                     & \textbf{-0.64**}                                                 &  & \textbf{0.02}                                               & -0.11                                                         \\
Combined training $\times$ female    & -0.10                                                  & -0.08                                                     & -0.32                                                            &  & 0.01                                                        & -0.14                                                         \\
\end{tabular}
\end{adjustbox}
\end{table}

\vspace{-5px}

 \section{Discussion and Conclusions}

\cite{Ubfal2022} conclude their analysis by discussing their findings, comparing them with those of other studies, and suggesting future research ideas to identify which business training designs perform best.

\vspace{-5px}

\subsection{Only Short-Run}

The whole discussion of the treatment effects found significant results only in the short term. Table \ref{sum-business-outcomes} finds positive impacts of the soft-skills training on profits and sales only after 3 months. Table \ref{sum-mechanisms} illustrates the effects on the intermediate mechanisms and detects a substantially higher adoption of business practices only at the first follow-up. These findings contrast with those of \cite{Campos2017}, which in an RCT in Togo, providing a similar treatment (i.e., traditional business training versus personal initiative-oriented training), found persistent effects over a period of 2.5 years. \cite{Ubfal2022} motivate this difference with a number of reasons. First, the very unique cultural and institutional characteristics of the two countries may have interacted with the treatment. Second, \cite{Campos2017} provided monthly visits to the trainer after the training. It is clear that this additional support to entrepreneurs helped them maintain a high level of motivation and persistence. In this sense, \cite{Ubfal2022} suggests how probably relying solely on soft-skills training may not guarantee lasting improvements in business outcomes, prompting further research to compare, in a specific context, scenarios with and without customized follow-up interventions.

\vspace{-5px}

\subsection{Only Men}

Table \ref{sum-gender} provides evidence that although both men and women increase adoption of business practices, this translates into better business outcomes, particularly sales and profits, for men only. The experiment in Togo by \cite{Campos2017} finds significant and positive effects on profits for both genders, probably still due to follow-up visits. Nevertheless, \cite{Ubfal2022} point out that when the soft-skills index interacts with gender, it turns out to be half as high for women as for men. This discrepancy between practice adoption and outcomes might be explained by a potential influence of soft skills, where the stronger development of these skills among men might have enhanced the positive impact of business practices on their business outcomes, a boost that women did not experience to the same degree.

\vspace{-5px}

\subsection{Only Soft-Skills}

Not having enough personal initiative, perseverance, grit, and ability to cope with setbacks has a direct negative impact on business performance. In practice, from the results of \cite{Ubfal2022}, women who participated in soft-skills training adopted more of the recommended business practices, but the soft-skills index showed that they did not significantly develop their soft-skills, resulting in no effect on business outcomes. Furthermore, throughout the analysis, the results for combined training appeared mostly insignificant or much lower than for soft-skills training. All in all, a robust conclusion of \cite{Ubfal2022} is the demonstration of the greater effectiveness of soft-skills training than combined-skills training.